\section{Chapter Summary}

This chapter outlined the value of computational pipelines as a means to automate data processing in sport. In particular, it highlighted that data provenance, the ability to trace the results of an analysis back to the original input, is still an open research area.

While computational pipelines are well established in scientific fields such as bioinformatics, this thesis brings the benefits of computational pipelines to sport analysis. To assist with this process, this chapter proposed a specialised notation (\secref{sec:ournotation}) for tracking data provenance in the sport domain based on W3C PROV.

While designing a software tool that uses this notation to help annotate data and automate workflows is a future possibility, even as-is the notation can be used to help document workflows. The notation is used in \chref{ch:integration} to describe the pipeline developed in this thesis.

\pagebreak{}

\subsection*{Future Work}

%This chapter evaluated functionality of existing systems against the motivating scenario provided in \secref{sec:provmotivation} and systematically analysed existing notations against Moody's Physics of Notations \cite{Moody2009}.

Future work is needed to formally interview sport performance analysts to verify whether the motivating scenario is reflective of their real-world experience, and to develop tool support for the provenance notation so that users can more easily utilise the notation to document workflows. The notation has been utilised in this thesis (\chref{ch:integration}), but trials are needed with sport performance analysts and sport researchers to evaluate the usability of the proposed notation and to determine whether it fulfils their needs.

\subsection*{Contributions}

\begin{enumerate}
  \item Provided an analysis of the data provenance needs of the sport domain, and evaluated existing data provenance tools against these criteria. A customised data provenance notation for sport was proposed in order to ease uptake for sport performance analysts without a computer science background.
\end{enumerate}
