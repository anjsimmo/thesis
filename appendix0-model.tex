\openchapterblock

%\chapter{Data Provenance for Sport}
\chapter{Modelling}
\label{appendix:modelling}

\section{Application of Abstract Data Model}

\subsection{Concrete Syntax}\label{concrete-syntax}

This section utilises the abstract data model to define concrete data models
that describe the schema and meta-data of real-world data. This can be performed
directly within a general purpose programming language, or through definition of a
Domain Specific Language (DSL) based on the abstract model to ease definition of the
concrete models. In respect to its ability to serve as language to describe concrete models,
the proposed abstract data model serves a similar purpose to a linguistic metamodel.
However, as K{\"{u}}hne notes in \textit{Matters of (meta-) modeling} \cite{Kuhne2006},
generalisation is a transitive relationship that directly applies to the end system,
and thus it is more appropriate to describe the model as ``abstract'' rather than as a true ``metamodel''.

For the purposes of this thesis, a very simple notation will be defined
for convenience:

\begin{itemize}
\item
  The syntax \emph{name} \texttt{:} \emph{type} will be used to annotate
  specific measurements with their type. For example,
  \texttt{ball\ position\ :\ Spatial}.
\item
  Dense parameters will be represented by postfixing them with a star
  \texttt{*}. For example, \texttt{time\ :\ Temporal*}.
\item
  The syntax \emph{attribute} \texttt{=} \emph{value} will be used to
  specify the meta-data of measured parameters. In the case of unknown
  parameters, the token \texttt{UNKNOWN} will be used.
\end{itemize}

\subsection{Examples}\label{data-collection}

These examples utilise the abstract data model and concrete syntax to
compare the data collection methods currently available in a range of
sports.

\subsubsection{Traditional Game Summary}\label{traditional-game-summary}

A human tallies events. Assume that official data is 100\% accurate.

% https://tex.stackexchange.com/questions/150965/insert-symbols-inside-verbatim-mode-latex/156456#156456
%http://mirror.aarnet.edu.au/pub/CTAN/macros/latex/contrib/listings/listings.pdf
\begin{lstlisting}[
  numbers=none,
  frame=none,
  mathescape=true,
  fontadjust=false,
  basicstyle=\ttfamily
]
Traditional Game Summary : Sensor Platform
  Human : Human Entry
    event : Event*
      Granularity = {goal, behind}
    tally : Count
      Accuracy = $\pm$0
\end{lstlisting}

\subsubsection{Database System}\label{database-system}

A human enters detailed game data based on video feeds. The error
estimates are taken from O'Shaughnessy \cite{oshaughnessy_possession_2006} and Jackson \cite{Jackson2016} who have both held positions at Champion Data, the AFL data provider.

% From Karl Jackson's PhD:
%   Champion Data has historically claimed an accuracy of 99% accuracy.
%   This has no basis through rigorous testing,
%   but it is assumed that it is close to the truth.
% (Ignore centrebounce macro and hyphenate here for consistency with other examples)
\begin{alltt}
Database System : Sensor Platform
  Human : Human Entry
    event : Event
      Accuracy    = 99%
      Granularity = \{kick, mark, handball, hit, tackle,
                     goal, behind, pick-up, centre-bounce,
                     throw-in, out-of-bounds, free-kick\}
    difficulty : Qualitative
      Inter-rate Reliability = UNKNOWN
      Granularity            = \{easy, hard\}
    ball position : Spatial
      Error Radius = 5-10 m
    time : Temporal
      error rate = 5 s
    player : ID
      accuracy = UNKNOWN
    event sequence order : ID*
      accuracy = UNKNOWN
\end{alltt}

\subsubsection{Computer Vision}\label{computer-vision}

Computer vision tracks two types of items: the ball, and the players. A
common issue with computer vision systems is losing track of the mapping
between trajectories and player identifiers when two players pass in close
proximity of each other.\footnote{The error estimates are taken from
  \url{http://www.stats.com/sportvu/sportvu-basketball-media/}}

\begin{verbatim}
Computer Vision : Sensor Platform
  Vision : Sensor
    time : Temporal*
      Accuracy    = UNKNOWN
      Sample Rate = 25 Hz
    item : ID*
      accuracy = UNKNOWN
    position : Spatial
      Error Radius : UNKNOWN
\end{verbatim}

\subsubsection{Wearable Tracking Device}\label{wearable-tracking-device}

Unlike computer-vision, wearable devices ensure that the player identifier is
always 100\% accurate as it is physically attached to the player (unless
of course, if a player wears the wrong device).\footnote{The sample
  rates are taken from Catapult Sports circa 2015 (to reflect the technology available during the time-period of the primary dataset used in this thesis) and found in product information released to \url{https://web.archive.org/web/20170910220858/http://www.catapultsports.com:80/system/system/} and
  \url{https://youtu.be/F9ZsYEyf3HE?t=110}. GPS accuracy is taken from
  \url{http://www.ga.gov.au/scientific-topics/positioning-navigation/geodesy/geodetic-techniques/global-positioning-systems-gps/gps-consumer}}
Note that velocity vectors and orientation vectors are treated as
spatial data. Heart rate could be modelled in a number of ways; however, it was
decided to model it as a cumulative count, as this avoids ambiguities
that typically arise when describing variable rates.

\begin{verbatim}
Wearable Tracking Device : Sensor Platform
  GPS : Sensor
    time : Temporal*
      Accuracy    = UNKNOWN
      Sample Rate = 10 Hz
    player : ID*
      Accuracy = 100%
    position : Spatial
      Error Radius = 3 m
    velocity : Spatial
      Error Radius = UNKNOWN
  Accelerometer : Sensor
    time : Temporal*
      Accuracy    = UNKNOWN
      Sample Rate = 1000 Hz
    player : ID*
      Accuracy = 100%
    acceleration vector : Spatial
      Error Radius  = UNKNOWN
  Gyroscope : Sensor
    time : Temporal*
      Accuracy    = UNKNOWN
      Sample Rate = 1000 Hz
    player : ID*
      Accuracy = 100%
    orientation vector : Spatial
      Error Radius  = UNKNOWN
  Heart Rate Monitor : Sensor
    time : Temporal*
      Accuracy    = UNKNOWN
      Sample Rate = UNKNOWN
    player : ID*
      Accuracy = 100%
    Cumulative Heart Beats : Count
      Accuracy = UNKNOWN
\end{verbatim}

% Not used in thesis
% \subsubsection{Eye Tracking}\label{eye-tracking}
%
% Eye tracking is another form of wearable technology. The gaze location
% error (if known) would be measured in radians.
%
% \begin{verbatim}
% Eye Tracking : Sensor Platform
%   Eye Goggles : Sensor
%     Gaze Location : Spatial
%       Accuracy = UNKNOWN
%     time : Temporal*
%       Accuracy    = UNKNOWN
%       Sample Rate = UNKNOWN
%     player : ID*
%       Accuracy = 100%
% \end{verbatim}

\closechapterblock
