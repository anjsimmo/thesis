
\section{Chapter Summary}

This chapter brought the components presented in previous chapters together into a platform for spatio-temporal sport analysis. It demonstrated the value of the platform as a means to rapidly form hypotheses, quantitatively investigate team formations, and to facilitate tracing results back to the underlying data to ensure all sport practitioners can understand and audit results of the analysis. While some of the patterns found in this chapter were only weakly statistically significant, and limited by the observational nature of sport analysis, the results can still be insightful for sport practitioners, and can serve as a basis upon which to form hypotheses to test through trialling strategies in real games.
% (left as future work).

%This chapter integrated the components described in the thesis through a study of team shape.

\pagebreak{}

This work is distinct from prior work in that:

\begin{enumerate}
   \item The analysis is powered by only non-identifiable data. This facilitates ethical data sharing.
   \item The player tracking GPS data input is expressed as latitude, longitude points, which are then transformed to the coordinate system of the field. This allows any GPS tracking device to be used, making the technique accessible to sub-elite teams who cannot afford a local positioning system.
   \item All analysis results can be traced back to the underlying data, thus allowing sport practitioners to investigate interesting patterns identified in greater detail and to develop trust in the system through an ability to scrutinise the results.
\end{enumerate}

%\subsection*{Threats to Validity}
\subsection*{Limitations}

%While the data processing pipeline was automated such that it could operate on any data or playing venues,

The analysis in \secref{sec:shapepaper} was limited to five home matches. While each match contains many events, the five home matches selected may not be representative of other matches. In particular, they may be limited by player availability, and the playing style may differ against different opponents. Moreover they were all from the perspective of the same club (Geelong Football Club) and played at the same venue (Kardinia Park).

\subsection*{Future Work}

From a research perspective, future work is needed to run the analysis on a larger number of matches, and to work with other \afl{} teams. The pipeline constructed in this chapter will process data for any matches available, and (due to the re-projection step) allows merging of match data from multiple venues. Thus while this thesis has solved the technical challenges involved, running a larger study will require development of a collaboration approach to unite all stakeholders involved.

From an applied perspective, future work is needed to translate the proof of concept developed in this thesis into a production ready tool that sport analysts can apply. While each \textit{component} has been designed to consider the usability needs of sport performance analysts, the system \textit{as a whole} is not currently in a state that sport performance analysts could install the software and use it themselves. Thus additional engineering effort is needed to simplify the deployment and improve the consistency of the user interface. Furthermore, while de-identification is important when sharing data outside the club, internally the analysis is of most value when the club can trace the results to the individual player level so they can deliver personalised feedback. Thus there is a need for clubs to be able to disable the data de-identification step when running analysis for their own internal use (i.e. without randomising player order and without downsampling the data).

As the system aims to provide a \textit{platform} for analysis, rather than limiting sport performance analysts to just the analysis procedures demonstrated in \secref{sec:shapepaper}, future work is needed to design an end-user programming environment that will support sport performance analysts to build new functionality themselves. The analysis procedures for \secref{sec:shapepaper} were implemented as Python scripts within a Jupyter notebook\footnote{Software: Jupyter \url{https://jupyter.org/} Accessed: 2019-11-25} that operated upon the data generated by the processing pipeline. Future work could develop a Domain Specific Visual Language (DSVL) to assist sport performance analysts to graphically express queries over the refined dataset in order to alleviate the need to learn Python. The notation developed in \chref{ch:prov} could be utilised as part of the language as a way to assist sport performance analysts to construct deeper analysis pipelines that build further upon each other.

Finally, future sport science research is needed to build upon and extend the platform. For example, extending the analysis to multiple teams to study interactions, and to evaluate the information gain from incorporating team spread into the analysis. This is elaborated further in \secref{sec:conclusion-future-work}.

\subsection*{Contributions}

\begin{enumerate}
  \item Developed a platform for spatio-temporal analysis of team sport, demonstrated through AFL as an example. Feedback from an elite AFL club confirms that the techniques proposed are likely to offer useful insights.
  \item Performed an analysis of team shape and game speed in AFL. Team spread was found to strongly correlate with game speed.
\end{enumerate}
